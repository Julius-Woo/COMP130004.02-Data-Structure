\documentclass[UTF8]{ctexart}
\ctexset { section = { format={\Large \bfseries } } }
\pagestyle{plain}
\usepackage{float}
\usepackage{amsmath}
\usepackage{amssymb}
\usepackage{listings}
\usepackage{graphicx}
\usepackage{xcolor}
\usepackage{geometry}
\geometry{a4paper,scale=0.8}
\usepackage{caption}
\usepackage[linesnumbered,ruled]{algorithm2e}
\captionsetup[figure]{name={Figure}}

\lstset{
language=Python, % 设置语言
basicstyle=\ttfamily, % 设置字体族
breaklines=true, % 自动换行
keywordstyle=\bfseries\color{blue}, % 设置关键字为粗体,
morekeywords={}, % 设置更多的关键字,用逗号分隔
emph={self}, % 指定强调词,如果有多个,用逗号隔开
emphstyle=\bfseries\color{Rhodamine}, % 强调词样式设置
commentstyle=\color{black!50!white}, % 设置注释样式,斜体,浅灰色
stringstyle=\bfseries\color{red!90!black}, % 设置字符串样式
columns=flexible,
numbers=left, % 显示行号在左边
numbersep=2em, % 设置行号的具体位置
numberstyle=\footnotesize, % 缩小行号
frame=single, % 边框
framesep=1em % 设置代码与边框的距离
}

\title{\textbf{Data Structure lab6}}
\author{吴嘉骜 21307130203}
\date{\today}

\begin{document}

\maketitle

\noindent
\textbf {\large Objective}\\  The objective of this lab is to understand the spirit of Dynamic Programming and to apply it to solve problems.\\
\noindent
\textbf {\large Experiment environment} \\
   Windows 11 VsCode Python 3.11.5 64-bit\\

\setlength{\parindent}{0pt}
\textbf{\large Task description}\\
Suppose you have one machine and a set of $n$ jobs $a_1, a_2, \ldots , a_n$ to process
on that machine. Each job $a_j$ has a processing time $t_j$, a profit $p_j$, and a
deadline $d_j$. The machine can process only one job at a time, and job $a_j$ must
run uninterruptedly for $t_j$ consecutive time units. If job $a_j$ is completed by its
deadline $d_j$, you receive the profit $p_j$, but if it is completed after its deadline,
you receive the profit of 0.

Give an algorithm to find the schedule that obtains the maximum amount of
profit, assuming that all processing times are integers between 1 and n. Please
write code and analyze time complexity.
\section{Algorithm}
We use a dynamic programming (DP) approach to solve the job scheduling problem for maximum profit.\\
First we sort the jobs by their deadlines. This is because it makes sense to consider earlier deadlines first for scheduling, and it will be useful when we build up the DP table next.\\
We then construct a 2D DP table $dp$ with size $(n+1) \times (d_n+1)$, where $n$ is the number of jobs and $d_n$ is the maximum deadline. 
The $dp(i,j)$ element represents the maximum profit that can be obtained by scheduling jobs $1,2, \ldots, i$ adhering to deadlines, within total time $j$, or $j$ slots.
Note that $i=0$ represents the case where there are no jobs to schedule, and $j=0$ represents the case where there is no time slot to schedule the jobs.
Both cases result in a profit of 0.\\

For $dp(i,j), i \geq 1$, consider whether we want to schedule job $i$ or not:\\
If we do not schedule job $i$, then the maximum profit is the same as scheduling jobs $1,2, \ldots, i-1$ within $j$ slots, which is $dp(i-1,j)$.\\
If we schedule job $i$, assume that the latest time slot that $a_i$ can start is $t'$, then it should satisfy $t' \leq j-t_i$ and $t' \leq d_i-t_i$, which leads to
$t' = min\{j, d_i\}-t_i$.\\ When $t' \geq 0$, we claim that $dp(i,j) = p_i + dp(i-1, t')$. Fisrt notice that $a_i$ is the job with the latest deadline among jobs $1,2, \ldots, i$, and we
always arrange it at the latest possible time slot. If we schedule $a_i$ ahead of $t'$, say $t'' < t' $,
the profit will be $p_i + dp(i-1, t'') \leq p_i + dp(i-1, t')$, since $dp(i-1,t'') \leq dp(i-1, t')$.\\
Second, we prove that $p_i + dp(i-1, t')$ maximizes the profit. As shown above, if $p_i+ dp(i-1, t'')$ may achieve a higher profit, then $t'' \geq t'$. As $t'$ is the latest possible time slot for $a_i$,
there must be no jobs in $1,\ldots,i-1$ scheduled after $t'$, which means that we can use $dp(i-1, t')$ to replace the profit achieved before $t''$. 
Therefore, $p_i + dp(i-1, t')$ is the maximum profit.\\
Conversely, When $t'<0$, it means that to arrange $a_i$ we have to start it before time slot 0, which is impossible. In this case, $dp(i,j) = dp(i-1,j)$.\\

In conclusion, we have the following recurrence relation:\\
\begin{equation*}
   dp(i,j) = 
\begin{cases} 
dp(i-1,j) & \text{if } t' < 0 \\
\max\{dp(i-1,j), p_i + dp(i-1,t')\} & \text{if } t' \geq 0 \\
\end{cases}
\end{equation*}

When the dp table is filled, the maximum profit is $dp(n,d_n)$, and we can trace back to find the schedule. We start from $dp(n,d_n)$, and if $dp(n,d_n) \neq dp(n-1,d_n)$, it means that $a_n$ is scheduled.
Add $a_n$ to the schedule, and move to $dp(n-1,d_n-t_n)$. If $dp(n,d_n) = dp(n-1,d_n)$, it means that $a_n$ is not scheduled, and we move to $dp(n-1,d_n)$. Repeat this process until we reach $dp(0,0)$.\\

The pseudocode for the algorithm is as follows:

\begin{algorithm}[H]
   \SetAlgoLined
   \KwResult{Maximum profit and job schedule}
   \KwIn{List of jobs with processing time, profit, and deadline}
   Sort jobs by deadline\;
   Initialize DP table with zeros\;
   \For{each job \(i\) from \(1\) to \(n\)}{
     \For{each time \(j\) from \(1\) to latest deadline}{
       Compute the latest start time $t'$ for job \(i\)\;
       \eIf{$t'<0$, job \(i\) cannot be started}{
         DP value is the same as without job \(i\)\;
       }{
         DP value is the max of without job \(i\) or with job \(i\) plus its profit\;
       }
     }
   }
   Trace back to find the schedule\;
   \Return{maximum profit and schedule}\;
   \caption{Maximize Profit Schedule Algorithm}
\end{algorithm}

\newpage
\section{Coding and Experiment}
The implemented code in \texttt{jobschedule.py} is shown below:\\
\begin{lstlisting}
import numpy as np
def max_profit_schedule(jobs):
      '''
      Find the schedule to maximize the profit given a list of jobs.
      
      Parameters:
         - jobs: a list of jobs, where each job is a tuple of (processing time, profit, deadline)
      
      Returns:
         - max profit: the maximum profit that can be obtained by scheduling the jobs
         - schedule: a list of job indexes that gives the maximum profit (in chronological order)
      '''
      # Sort jobs according to deadlines
      jobs.sort(key=lambda x: x[2])
      n = len(jobs)
      ddl_latest = jobs[-1][2]    # latest deadline
      dp = np.zeros((n+1, ddl_latest+1), dtype=int)  # dynamic programming table
      
      for i in range(1,n+1):  # i: job index+1
         for j in range(1, ddl_latest+1):  # j: deadline
            ti = jobs[i-1][0]
            pi = jobs[i-1][1]
            di = jobs[i-1][2]
            tmp = min(di, j) - ti  # tmp: the latest time to start job i
            if tmp < 0:  # cannot start job i
                  dp[i][j] = dp[i-1][j]
            else:  # can start job i
                  dp[i][j] = max(dp[i-1][j], dp[i-1][tmp] + pi)  # max profit of job i
      
      max_profit = dp[n][ddl_latest]
      # Trace back through the dp table to find the jobs that were scheduled
      schedule = []
      j = ddl_latest
      for i in range(n,0,-1):
         if dp[i][j] != dp[i-1][j]:  # job i-1 is scheduled
            schedule.append(i-1)
            j = j - jobs[i-1][0]  # move to the time slot before the start of this job
      schedule.reverse()
      return max_profit, schedule
\end{lstlisting}

We test the code with the given test cases and the results are shown below:\\
\begin{lstlisting}
   # test
sample_jobs = [
   [(2, 60, 3), (1, 100, 2), (3, 20, 4), (2, 40, 4)],
   [(3, 100, 4), (1, 80, 1), (2, 70, 2), (1, 10, 3)],
   [(4, 100, 4), (2, 75, 3), (3, 50, 3), (1, 25, 1)],
   [(2, 60, 3), (1, 100, 2), (3, 20, 3), (2, 40, 2), (2, 50, 3)],
   [(2, 60, 3), (1, 100, 2), (3, 20, 4), (2, 40, 4), (2, 50, 3), (1, 80, 2)],
   [(2, 60, 3), (1, 100, 2), (3, 20, 3), (2, 40, 2), (2, 50, 3), (1, 80, 2), (4, 90, 4)],
   [(3, 60, 3), (2, 100, 2), (1, 20, 2), (2, 40, 4), (4, 50, 4)],
   [(2, 60, 3), (1, 100, 2), (3, 20, 3), (2, 40, 2), (4, 50, 4), (1, 80, 2), (4, 90, 4)],
   [(3, 60, 3), (2, 100, 2), (1, 20, 2), (2, 40, 2), (4, 50, 4), (5, 70, 5)],
   [(2, 60, 3), (1, 100, 2), (3, 20, 3), (2, 40, 2), (4, 50, 4), (5, 70, 5), (3, 90, 4)]
]
for jobs in sample_jobs:
   max_profit, schedule = max_profit_schedule(jobs)
   print(max_profit, [jobs[idx] for idx in schedule])

# Output:
160 [(1, 100, 2), (2, 60, 3)]
180 [(1, 80, 1), (3, 100, 4)]
100 [(1, 25, 1), (2, 75, 3)]
160 [(1, 100, 2), (2, 60, 3)]
220 [(1, 100, 2), (1, 80, 2), (2, 40, 4)]
180 [(1, 100, 2), (1, 80, 2)]
140 [(2, 100, 2), (2, 40, 4)]
180 [(1, 100, 2), (1, 80, 2)]
100 [(2, 100, 2)]
190 [(1, 100, 2), (3, 90, 4)]
\end{lstlisting}

\section{Analysis}
The initial sorting of the jobs takes $O(n\log n)$ time.\\
The main loop takes $O(n \cdot d_n)$ time, where $d_n$ is the maximum deadline,
since all the operations inside the loop take constant time.\\
The trace back loop takes $O(n)$ time, since it traces back at most $n$ steps.\\
Then the total time complexity is $O(n\log n + n \cdot d_n)$.\\
When $d_n$ is larger than $n$, the time complexity is $O(n \cdot d_n)$ since $n \cdot d_n$ dominates $n\log n$.\\
When $d_n$ is smaller than $n$, the time complexity can also be reduced to $O(n \cdot d_n)$ because sorting $n$ integer numbers in range $[1,d_n]$
can be implemented in $O(n+d_n)$ time using counting sort.\\
So, the time complexity is $O(n \cdot d_n)$.\\

The space complexity is also $O(n \cdot d_n)$, since the DP table has size $(n+1) \times (d_n+1)$. If we don't
want the schedule, we can reduce the space complexity to $O(d_n)$ by only keeping the previous row of the DP table.\\
\end{document}